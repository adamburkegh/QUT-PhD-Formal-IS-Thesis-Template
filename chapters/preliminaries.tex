A preliminaries chapter lays out formal definitions from existing literature used throughout the text.

This template has been used to publish the thesis of Buijs~\cite{MScBuijs2010} and is originally used for the thesis of Nugteren~\cite{MScNugteren2010}. 

One of the best resources for \LaTeX basics, and advanced constructs, is the \LaTeX wikibook\footnote{To be found at~\url{http://en.wikibooks.org/wiki/LaTeX/}}. Of course colleagues and a good internet search using your favorite search engine can do wonders if you're stuck. 

Below is an example section and definition, to show how the glossary works.

\section{Activities and Languages}
\label{sec:prelim:actlang}

An activity is an identifiable thing that happens in a process. 

\begin{definition}[Activities and Traces]
\label{defn:activities}
Let $\activities$ be a set of activities in a process, and $\traces$ the possible finite sequences of those activities. A \emph{trace} $\sigma \in \traces$ is a sequence of activities. 
\end{definition}